\chapter{An Introduction to BGP}

As the Internet grew it was no longer possible to keep all routers in a single routing domain.
It was necessary to add another layer of organization.
All the router that belonged to the same organization and where controlled by the same network administration team were grouped in an Autonomous System.
For example, Universitat Pompeu Fabra belongs to autonomous system Red Iris with Autonomous System Number AS766.
Guifi.net is a community network that has AS49835.

Within an autonomous system, the system administrators configure an Interior Gateway Protocol IGP.
Examples of IGP are RIP, OSPF and IS-IS.
The network administration team agrees in the protocol that is going to be used and the per-hop metrics.
Then the protocol computes the routes that minimize those metrics.

There is another layer of routing between Autonomous Systems.
When a network has a packet destined to another autonomous system, it needs to decide to which neighbouring autonomous system tha packet will be forwarded.

The protocol that is used in the Internet to route between Autonmous Systems is the Border Gateway Protocol (BGP).
A key difference between IGPs and BGPs is that the latter routes packets between different organizations that not necessarily trust each other nor share the same goals.
Choosing the shorter path is not always the best option, and it is not possible in agreeing in common metric for each connection between two ASs.
If two alternative paths exist, one AS may prefer one path and the other AS the other path.

There are ASs of many different sizes and scopes.
Some ASs are local and provide access to end users, while others are international and only provide services to other ISPs.
An example of the former is Jazztel and an example of the latter is Cogent.

The connections between ISPs can be of two different types: either transit or peering.
In the peering relation, two ISPs exchange traffic without charging each other.
In the case of transit, there is one ISP that is a customer of the other and pays for the carried traffic.

\begin{figure}[!h]
\centering
\includegraphics[width=\linewidth]{figures/peering-and-transit.eps}
\caption{Example network topology with Autonomous Systems and peering and transit connections}
\label{fig:peering-and-transit}
\end{figure}

Fig.~\ref{fig:peering-and-transit} shows an example topology in which clouds represent autonomous systems that are connected with transit and peering links.
Different autonomous systems normally connect at Internet exchange points. 
Catnix is one example of an Internet exchange point.

In routing between different ASs, it is necessary to account for policy.
Each ASs has a different policy that is translated to routing announcements and routing decisions. 

For example, if C has a packet for the AS G, the shortest path is via AS A.
However, AS C has to pay for the transit that is sent to A, and therefore C prefers to send the packet to AS D with which it has a peering agreement.
In this case, policy is taken into account to make a routing decision.

An example of route announcement conditioned by policy is that AS D does not announce a route to AS E to AS C even though this route exists.
The reason is that AS D is not being payed for carrying transit from AS C to AS E and in principle has no incentive to carry that traffic.

With BGP both route announcements and routing decisions can be conditioned by policy.
In principle, the order of preference for routing a packet is 
\begin{itemize}
\item to a client (the AS makes money)
\item to a peer
\item to a provider (the AS pays money)
\end{itemize}

Regarding the announcements, the normal policy is
\begin{itemize}
\item announce the routes of clients to everyone.
\item announce the routes of peers to clients.
\item announce the routes of providers to clients.
\end{itemize}

It is possible to also set policy in the form of BGP filters.
Back in 2008, Pakistan telecom hijacked Youtube traffic by announcing a more specific prefix.
This announcement was not filtered.
Instead it was filtered and propagated by other AS effectively redirecting Youtube traffic to Pakistan Telecom.

The following is a list of events\footnote{quoted from http://www.circleid.com/posts/ 82258\_pakistan\_hijacks\_youtube\_closer\_look}
\begin{quote}
\begin{itemize}
\item 18:47:45    First evidence of hijacked route propagating in Asia, AS path 3491 17557
\item 18:48:00    Several big trans-Pacific providers carrying hijacked route (9 ASNs)
\item 18:48:30    Several DFZ providers now carrying the bad route (and 47 ASNs)
\item 18:49:00    Most of the DFZ now carrying the bad route (and 93 ASNs)
\item 18:49:30    All providers who will carry the hijacked route have it (total 97 ASNs)
\item 20:07:25    YouTube, AS 36561 advertises the /24 that has been hijacked to its providers
\item 20:07:30    Several DFZ providers stop carrying the erroneous route
\item 20:08:00    Many downstream providers also drop the bad route
\item 20:08:30    And a total of 40 some-odd providers have stopped using the hijacked route
\item 20:18:43    And now, two more specific /25 routes are first seen from 36561
\item 20:19:37    25 more providers prefer the /25 routes from 36561
\item 20:28:12    Peers of 36561 start seeing the routes that were advertised to transit at 20:07
\item 20:50:59    Evidence of attempted prepending, AS path was 3491 17557 17557
\item 20:59:39    Hijacked prefix is withdrawn by 3491, who disconnect 17557
\end{itemize}
\end{quote}

Pakistan Telecom advertised a prefix (208.65.153.0/24) that was more specific that the one advertised by Youtube (208.65.152.0/22) and therefore it was used by BGP routers.
The response of Youtube was to announce two even more specific routes /25.
