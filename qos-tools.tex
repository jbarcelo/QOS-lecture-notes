\chapter{QoS Tools}
\section{Why QoS tools?}
In the last chapter we revised different applications with heterogeneous QoS requirements.
If all these applications coexist in the same network, it is necessary to make sure that they all see their requirements fulfilled.

One possible alternative is bandwidth overprovisioning.
If there is more than enough bandwidth, the queues are always empty and the packets suffer minimum delay and packet loss.

One of the problems of overprovisioning is that sometimes bandwidth can be very expensive.
One of the challenges of the newtwork engineers is to satisfy the QoS requirements at a reasonable cost.

Another problem of overprovisioning is the elastic demand.
Imagine that you overprovision a network to make sure that the queues are empty and the VoIP packets suffer no queueing delays.
Some users will notice that the network is blazing fast.
``Hey, I can download a HD movie in no time, I will download many of them so I can choose which one I want to watch.''
It is normal that users respond to bandwidth availability by consuming extra bandwidth.

Finally, and very similar to the previous case, there are network security threats such as worms that will consume as much bandwidth as it is available to propagate.
Sometimes a defective or misconfigured device will also consume as much bandwidth as it is available.

For all these reasons, overprovisioning cannot be the answer to everything.
An alternative to overprovisioning is to use QoS tools to manage the available bandwidth in such a way the different QoS requirements can be met by prioritizing one traffic over the others and shaping the traffic flows.

Using the road analogy, if we have an ambulances with strict delay constraints we can make the roads wider or use 


