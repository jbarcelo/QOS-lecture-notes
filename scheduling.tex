\chapter{Scheduling}

Scheduling is a key tool in QoS and it is complex enough to deserve its own chapter.
A scheduler serves two or more queues and has to take decissions about which queue has to be served first.
The combination of queues and scheduler can change the order of the packets.
For this reason it is important that all packets of a same flow are mapped to the same queue, to prevent packet re-ordering.

\section{Strict Priority Queues}

Prioritizing is a core idea in QoS.
Imagine that a big packet belonging to backup tool such as Dropbox arrives to a router.
Right after that, a secon packet arrives to the same roter.
Imagine that this second packet is a small VoIP packet.

The VoIP packet is in a hurry, because it must make it to the decoder before the playout time.
For this reason, the combination of queues and scheduler will allow the VoIP packet to advance the backup packet and be transmitted in the first place.

In the simplest case we have only two queues: the high priority queue and low priority queue.
A classifier maps each of the packets to one of the queues.
The scheduler will serve the high priority whenever there is a packet in that queue.
Only when the high priory queue is empty, the scheduler will serve the low priority queue.

Note that this concept can be easily generalized to the case in which there are more than two queues.
The principle is that a queue will be served only when all the higher priority queues are empty.

\subsection{Preemptive strict priority}
In preemptive strict priority, if a high priority packet arrives while a low priority packet is being served, the service of the low priority queue is interrupted and the high priority packet is served immediately.
Theoreticall, the service of the low priority packet is resumed when the high priority queue becomes empty.

This approach is very benefitial for low priority packets as they are never disturbed by low priority packets.
The only problem with this approach is that, in practice, it is not trivial to stop a transmission of a packet and later resume it.

\subsection{Non-preemptive stric priority}
In non-preemptive strict priority, if a high priority packet arrives while a low priority packet is being served, the transmission is not interrupted.
The high priority packet will wait for the transmission of the low priority packet to finish and then it will be transmitted.

This is the approach that is used in practice in high speed transmission lines.
The problem with this solution is that, in a slow transmission line, a high priority packet might need to wait for a long time if it arrives while a long low priority packet starts being serviced.

\subsection{Fragmenting and interleaving}
A 
