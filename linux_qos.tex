\chapter{Linux QoS}

In this chapter we will overview the queues and QoS tools available in linux.
\begin{figure}[!h]
\centering
\includegraphics[width=\linewidth]{figures/linux_qos.eps}
\caption{Flow of packets through different layers and queues.}
\label{fig:linux_qos}
\end{figure}

As shown in Fig.~\ref{fig:linux_qos}, a linux box transmits two kinds of packets.
On the one hand we have the packets coming from local applications, such as a web browser.
On the other hand, there is traffic that has been received from another interface and it is being forwarded towards the final destination.
The second kind of traffic is present only if the computer is working as a router.

The IP stack places the packets into the \emph{queueing discipline} layer.
It is in this layer in which linux can apply QoS tools to the outgoing traffic.
Then, the packets are sent to the driver queue (or transmission ring queue), which is a simple FIFO queue in which there is no chance of using QoS tools.
Finally, the network interface card takes packet

