\chapter{Advertisement}

In this chapter we reproduce a \emph{technical specification} from \emph{BT}. 
It illustrate the kind of offers and jargon used by \emph{ISP}s.
The goal is that, by the end of the course, you have a good understanding of what this offer means and how the \emph{ISP}s internally engineer their networks to be able to provide the advertised services.

\section{BT MPLS}
        
        
\subsection{BT MPLS technical specification}

We're striving for our high-speed, fully secure IP VPN platform to set the standard for others to follow. Our BT MPLS (Multi-Protocol Label Switching) service over a Cisco powered network offers you the following main advantages:

\begin{itemize}
\item High-performance packet switching along pre-determined paths.
\item Independence of the underlying open systems interconnection (OSI) layer 2 connectivity.
\item High scalability enabling rapid network expansion.
\item Segregation of client VPN traffic from others by the use of unique route distinguishers.
\item Retention of your existing IP address scheme.
\end{itemize}

BT MPLS technology combines with Differential Services (DiffServ) and traffic engineering techniques to provide Quality of Service. It does so through the use of Priority Queuing Class Base Weighted Fair Queuing, Committed Access Rate, Weighted Random Early Detection (WRED), Frame Relay and ATM traffic shaping. This combination of MPLS and traffic engineering enables BT MPLS to differentiate between time critical, high priority traffic and delay tolerant, low priority traffic. You can classify and prioritise your applications into different Classes of Service, matching network performance with business need. MPLS offers a similar level of in-built security to existing Frame Relay and ATM networks.

More mechanisms to manage the performance of all applications on your network are available when you combine BT MPLS with the Application Optimisation Service (AOS) of Application Assured Infrastructure (AAI). With AOS the performance objectives of each individual application can be set. The service dynamically assigns network resources to each application to ensure that your performance objectives are met.

\subsection{6 Class of Service (CoS) model}
BT MPLS' 6 Class of Service (CoS) model sets us apartin the global market. It is a world leading Class of Service proposition and addresses the growing trend of multiple application use by clients. This trend includes:

\begin{itemize}
\item Increasing use of multiple enterprise resource planning (ERP) applications to support critical business processes.
\item Greater use of, and interest in, IP telephony and video streaming applications, with LANs becoming more sophisticated.
\item Quality of Service (QoS) mechanisms being introduced in client networks with DiffServ becoming the de facto standard for CoS implementations.
\end{itemize}

The 6 Class of Service Model gives you the ability to enjoy a greater granularity in bandwidth prioritisation and partitioning. You can choose to prioritise mission critical applications such as Siebel, SAP, Oracle and Lotus Notes into distinct prioritisation data classes. This enables you to run multiple applications simultaneously. In addition, you are able to run multimedia applications as well as IP Voice, enabling you to gain a high level of convergence on a global platform.

The 6 CoS model provides high flexibility and scalability, enabling you to easily burst between classes without the need for complicated configurations. Changes to your network can be kept to a minimum to support the new CoS model with end to end transparency ensuring configurations occur easily and swiftly across multiple sites. With the use of the additional AOS of AAI you can maximise the performance of your network in accordance with your changing business priorities.

\subsection{The Offer}
The CoS model uses DiffServ Code Point. We have developed the BT MPLS CoS model with clients and their application providers.

The 6 CoS model contains:
\begin{enumerate}
\item Voice - such as VoIP and PSTN breakout.
\item Assured Data - ERP applications or Real-time Multimedia - such as video conferences and other interactive services.
\item Assured Data - ERP applications.
\item Assured Data - ERP applications.
\item Assured Data - ERP applications.
\item Standard - email, intranet, file transfer protocol (FTP), internet breakout telnet and other network management.
\end{enumerate}

\subsection{A communications solution for all}
BT MPLS is proving extremely popular with small and medium enterprises as well as multi-national organisations. We have more than 2,000 clients and 43,000 ports connected to our platform globally. The service is for you if you require:

\begin{itemize}
\item Performance optimisation of your network applications to improve processes.
\item Any-to-any connectivity - either through a meshed or partially-meshed network.
\item The ability to prioritise the transmission of different types of data applications.
\item Different performance and prioritisation levels for various application types.
\item A network that can expand and contract easily in line with your own expansion strategy.
\item LAN-to-LAN or WAN-to-WAN connectivity over private networks.
\end{itemize}

If you want to include dynamic optimisation for your network check the capabilities of our AOS from our AAI product.

BT MPLS is available in over 115 countries with full 6 Class of Service (Cos) capabilities, known as Native IP, and from more than 1200 Points of Presence (PoPs) in Europe, the Americas and the Asia Pacific Region.

\subsection{Class of Service network performance guarantees}
The BT MPLS 6 Class of Service (CoS) model addresses the growing need for multiple applications with different performance levels. In combination with AOS of AAI it provides you with increased visibility and control over your application performance. Current market trends include:

\begin{itemize}
\item Increased use of multiple ERP applications to support critical business processes.
\item Greater use of and interest in IP Telephony and video streaming applications.
\item Increasingly sophisticated LANs.
\item Quality of Service (QoS) mechanisms being introduced into client networks.
\item DiffServ becoming the de facto standard for CoS implementations.
\end{itemize}


\subsection{Application Class of Service}

\subsubsection{Voice}
An any-to-any IP service capable of delivering Voice over IP (VoIP)applications for those client swishing to construct "do it yourself" voice solutions. Underwritten by installation, availability, round trip delay, packet delivery and jitter service level agreements.
Application example : IP Voice, Unified Communications

\subsubsection{Multimedia}
Optimised to support realtime video applications, the class is underwritten by service level agreements on installation, availability, round trip delay and packet delivery. A maximum of four Assured Data classes are available with the Multimedia class.
Application example: Video conferencing, video streaming

\subsubsection{Assured Data (up to four applications or mix of multimedia applications)}
Up to four Assured Data classes are available for mission-critical or delay-sensitive data applications. Each critical data application can be assigned to a dedicated Assured Data class, enabling bandwidth to be dedicated to individual critical data applications. This prevents critical data applications competing for the same bandwidth and ensures the performance of each application. Assured Data classes are an any-to-any IP service underwritten by installation, availability, round trip delay and packet delivery service level agreements.
Application example:

Missioncritical, delay sensitive applications such as:
\begin{itemize}
\item Oracle
\item SAP
\item ERP
\item Data Warehousing
\item Critrix
\item Client Server
\item Multimedia applications
\end{itemize}

\subsubsection{Standard Data}
A basic any-to-any IP transport service for delay-tolerant data applications such as email or Intranet access. The Standard Data class is underwritten by installation, availability and round trip delay service level agreements.
Application example:

Delay tolerant applications such as:
\begin{itemize}
\item Internet/intranet browsing
\item File transfers
\item Other applications that can accommodate variable throughput delay
\end{itemize}


BT has an outstanding track record in implementing MPLS and provides solutions uniquely tailored to the requirements of your organisation. Clients as diverse as car hire companies and oil and gas service providers have benefitedfrom the successful introduction of BT MPLS.

With more than 20 years of providing VPN services to global businesses, and network coverage across five continents, BT is at the forefront of designing, implementing and managing global IP VPNs. As a key partner with Cisco, we were one of the pioneers of commercial MPLS services.

BT offers you comprehensive service level agreements and a single point of contact, in addition to our network design and service optimisation consultancy.

With BT MPLS, we can help you match your applications to Class of Service requirements, defining and optimising the routing protocol architecture, planning migration and troubleshooting inter-working problems.

The result is the kind of flexible and powerful network service that can help you enjoy business success and growth.


