\chapter{RSVP and MPLS-DiffServ-TE}

In this chapter we will present an overview of a set of technologies that offer a tight control on QoS.
This control comes at the price of complexity, but the ISPs and the manufacturers have decided that QoS is worth it.
This technologies offer control on the path followed by the packets, bandwidth reservation (and therefore guarantees) and fast re-route capabilities.

\section{Multi Protocol Label Switching (MPLS)}

In the IP paradigm, the routers forward the packets based on the destination IP of the packet.
MPLS offers a completely different alterative, in which the forwarding is not done using the destination address.
Instead, labels are used at each hop to decide the outgoing interface.
This approach is called ``virtual circuit packet network'' as it somehow emulates a circuit on a packet network, and it is opposed to the ``datagram'' IP paradigm.
The virtual circuit approach has some advantages and disadvantages compared to the datagram approach.

In the MPLS jargon, a virtual circuit is called a Label Switched Path (LSP).
Each of the packets has a header which is called ``the label''.
In fact, a packet may contain multiple labels and it is said that they are ``stackable''.
The last label can be found on top.

Upon reception of a packet, the outermost label it is inspected.
Each router has a lookup table indicating, for every incoming label, the outcoming label and the outcoming interface.
The rooter simply performs the lookup, pops the outermost label and pushes a new label before forwarding the packet.
To populate the lookup table, the LSP is established before starting forwading packets.

In MPLS, the routers are called Label Switch Routers (LSR) and the edge routers Label Edge Routers (LER).

MPLS is protocol agnostic in the sense that can carry any kind of packets.
Examples are IP packets and Ethernet packets.
This functionality can be used to create both Layer-3 and Layer-2 virtual private networks.

